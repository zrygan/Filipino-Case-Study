\section{Notation}

The following rules for notation will be used for the entirety of the paper to remove redundancy and notational ambiguity:

\begin{thm_rule} (\textbf{Languages and Alphabets})
    All \emph{languages} are represented by a script capital letter, while all \emph{alphabets} are represented by blackboard bold capital letter. For instance, the Filipino language is given by \(\mathcal{F}\) while the Filipino alphabet is given by \(\mathbb{L}\).
\end{thm_rule}

\begin{thm_rule} (\textbf{String Operations})
    The dot operator (\(\cdot\)) denotes a \emph{concatenation} of two strings, the plus operator (\(+\)) denotes the \emph{Kleene Plus}, the star operator denotes the \emph{Kleene Star}.
\end{thm_rule}

\begin{thm_rule} (\textbf{Regular Expressions})
    Regular expressions will use \emph{formal notation} and to say that some string \(s\) matches the regular expression modeled by \(\mathcal{R}\) then we denote this by \(s \in R\).
\end{thm_rule}

\begin{thm_rule} (\textbf{Special Characters in Regular Expressions})
    The symbol for the blank space or " " is \(\psi\), while the symbol for the empty string or "" is \(\lambda\).
\end{thm_rule}


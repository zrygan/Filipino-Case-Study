\section{Preliminaries}
\subsection{The Filipino Alphabet}

Let \(\mathcal{F}\) be the Alphabet for the Filipino language, this alphabet is
composed of 56 scripts and 11 punctuation marks \cite{OOP}. The 56 scripts are divided into
two, the first half being the capital letters of the modern Latin script with
the addition of "Ñ" and "Ng"; while the other half is the lower case variants
of each letter.

The 11 punctuation marks in the Filipino language are the: \textit{tuldok} (.),
\textit{tandang pananong} (?), \textit{tandang padamdam} (!), \textit{kuwit} (,),
\textit{kudlit} ('), \textit{gitling} (-), \textit{tutuldok} (:),
\textit{tuldok-kuwit} (;), \textit{panipi} ("), \textit{pambukas na panaklong}
((), \textit{pampasarang panaklong} ()), at ang \textit{tutuldok-tuldok} (...)

In mathematical notation, we can represent \(\mathcal{F}\) as the set:

\[
    \mathcal{F} = \{\text{a},\text{b},\dots,\text{z},\text{ñ},\text{ng},   \
    \text{A},\text{B},\text{C},\dots,\text{Z},\text{Ñ},\text{Ng}\}         \
    \cup \{\text{.},\text{?},\text{!},\text{,},\text{'},\text{-},\text{:}, \
    \text{;},\text{"},\text{(},\text{)}, \ldots\}
\]

and the size of \(\mathcal{F}\), \(\left|\mathcal{F}\right| = 67\).

We can also introduce subsets the following which are subsets of \(\mathcal{F}\).
\begin{enumerate}
    \item \(\mathbb{M}\) = \{.,?,!,,,',-,:,;,",(,),...\}, the set of punctuation marks
    \item \(\mathbb{V}\) = \{a,e,i,o,u,A,E,I,O,I\}, the set of upper and lower case vowels
    \item \(\mathbb{C}\) = \(\mathcal{F} - (\mathbb{M}\cup\mathbb{V})\),
          the set of upper and lower case consonants
\end{enumerate}



\subsubsection{Remarks on the Digraph: \text{ng/Ng} or "\textit{en dyi}"}

Although the letter "Ng" or "ng" is a concatenation of two separate graphemes or
symbols in \(\mathcal{F}\) (since \(\text{Ng} = \text{N}\cdot\text{g}\) and
\(\text{ng} = \text{n}\cdot\text{g}\)), the letter "Ng" is officially recognized
as a symbol in \(\mathcal{F}\) since it represents a distinct Filipino sound.
In particular, it represents the voiced velar nasal sound, or in the International
Phonetic Alphabet (IPA), the \textipa{N} sound \cite{Malabonga_2009}.

For instance, the word "hangin" has 5 letters namely: "h", "a", "ng", "i", "n",
since "ng" is pronounced as a velar nasal sound, not as two separate sounds
\textipa{n-g}. Take for instance the word "manger" where "ng" is a substring in
but is not pronounced as the velar nasal sound. Instead, its pronunciation is
\textipa{\textprimstress meI ndZ @r} ("meyn-jer"); not
\textipa{\textprimstress m\ae N Z @r} ("mang-jer"),
\textipa{\textprimstress m\ae N @r} ("mang-er") or
\textipa{\textprimstress m\ae N@r} ("manger").



\subsection{Common Errors}

\subsubsection{On Spelling}

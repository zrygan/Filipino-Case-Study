\section{Preliminaries}
\subsection{The Filipino Alphabet}

Let \(\mathcal{F}\) be the Alphabet for the Filipino language, this alphabet is
composed of 56 scripts and 11 punctuation marks \cite{OOP}. The 56 scripts are divided into
two, the first half being the capital letters of the modern Latin script with
the addition of "Ñ" and "Ng"; while the other half is the lower case variants
of each letter.

The 11 punctuation marks in the Filipino language are the: \textit{tuldok} (.),
\textit{tandang pananong} (?), \textit{tandang padamdam} (!), \textit{kuwit} (,),
\textit{kudlit} ('), \textit{gitling} (-), \textit{tutuldok} (:),
\textit{tuldok-kuwit} (;), \textit{panipi} ("), \textit{pambukas na panaklong}
((), \textit{pampasarang panaklong} ()), at ang \textit{tutuldok-tuldok} (...)

In mathematical notation, we can represent \(\mathcal{F}\) as the set:

\[
    \mathcal{F} = \{\text{a},\text{b},\dots,\text{z},\text{ñ},\text{ng},   \
    \text{A},\text{B},\text{C},\dots,\text{Z},\text{Ñ},\text{Ng}\}         \
    \cup \{\text{.},\text{?},\text{!},\text{,},\text{'},\text{-},\text{:}, \
    \text{;},\text{"},\text{(},\text{)}, \ldots\}
\]

and the size of \(\mathcal{F}\), \(\left|\mathcal{F}\right| = 67\).

We can also introduce subsets the following which are subsets of \(\mathcal{F}\).
\begin{enumerate}
    \item \(\mathbb{M}\) = \{.,?,!,,,',-,:,;,",(,),...\}, the set of punctuation marks
    \item \(\mathbb{V}\) = \{a,e,i,o,u,A,E,I,O,I\}, the set of upper and lower case vowels
    \item \(\mathbb{C}\) = \(\mathcal{F} - (\mathbb{M}\cup\mathbb{V})\),
          the set of upper and lower case consonants
    \item \(\mathbb{V}_\text{upper}\) is the set of upper case vowels
    \item \(\mathbb{V}_\text{lower}\) is the set of lower case vowels
    \item \(\mathbb{C}_\text{upper}\) is the set of upper case consonants
    \item \(\mathbb{C}_\text{lower}\) is the set of lower case consonants
    \item \(\mathbb{L}\) = \(\mathcal{F} - \mathbb{M}\), the set of consonants and vowels
\end{enumerate}

\subsubsection{Remarks on the Digraph: \text{ng/Ng} or "\textit{en dyi}"}

Although the letter "Ng" or "ng" is a concatenation of two separate graphemes or
symbols in \(\mathcal{F}\) (since \(\text{Ng} = \text{N}\cdot\text{g}\) and
\(\text{ng} = \text{n}\cdot\text{g}\)), the letter "Ng" is officially recognized
as a symbol in \(\mathcal{F}\) since it represents a distinct Filipino sound.
In particular, it represents the voiced velar nasal sound, or in the International
Phonetic Alphabet (IPA), the \textipa{N} sound \cite{Malabonga_2009}.

For instance, the word "hangin" has 5 letters namely: "h", "a", "ng", "i", "n",
since "ng" is pronounced as a velar nasal sound, not as two separate sounds
\textipa{n-g}. Take for instance the word "manger" where "ng" is a substring in
but is not pronounced as the velar nasal sound. Instead, its pronunciation is
\textipa{\textprimstress meI ndZ @r} ("meyn-jer"); not
\textipa{\textprimstress m\ae N Z @r} ("mang-jer"),
\textipa{\textprimstress m\ae N @r} ("mang-er") or
\textipa{\textprimstress m\ae N@r} ("manger").

\subsection{Common Errors}

\subsubsection{Loan Words, U/O, and I/E}

Given the Spanish and English roots of Filipino, some \textit{loan} words have
rules for Filipino spelling. Let \(s\) be any string, the English language
\(\mathcal{E}\), the Spanish language \(\mathcal{S}\), and \(\mathcal{F}(s)\) is
the translation of \(s\) in \(\mathcal{F}\)

\begin{enumerate}
    \item \(s\in \mathcal{S} \rightarrow \mathcal{F}(s) : \text{ es}\cdot
          (\mathbb{M}|\mathbb{C})* \)
    \item \(s\in \mathcal{E} \rightarrow \mathcal{F}(s) : \text{ is}\cdot
          (\mathbb{M}|\mathbb{C})* \)
\end{enumerate}

Rule (1) denotes that if \(s\) is a Spanish word, translating \(s\) to a Filipino
word would use "es" as the prefix to the word to denote that \(\mathcal{F}(s)\)
is a word of Spanish origin. On the other hand, for rule (2), if \(s\) is an
English word, then \(\mathcal{F}(s)\) would use "is" as the prefix to the word
to denote that it is of English origin.

Another rule for Spanish loan words is the "o" to "u" change. If \(s\in \mathcal{S}\)
was translated to \(\mathcal{F}\), and the the prefix of \(s\) is given by the
regular expression \((\text{C}|\text{c})\text{on}(\text{f}|\text{v})\). Then,
\(\mathcal{F}(s)\) is prefixed with
\((\text{C}|\text{c})\text{um}(\text{p}|\text{b})\). In particular, if the
\(s \models \text{conf} \rightarrow \mathcal{F}(s) \models \text{kump}\),
otherwise \(\mathcal{F}(s) \models \text{ kumb}\).

In addition to this, let \(\alpha\) be the prefix, \(\omega\) be the suffix, and
\(s\in\mathcal{F}\) be any Filipino word and \(s \models \mathbb{(M|C)}+(\text{e}|\text{o})\).
If another word \(k\in\mathcal{F}\) and \(k=\alpha\cdot s\cdot\omega\), then
\(k\models\alpha\mathbb{(M|C)}+(\text{i}|\text{u})\omega\). That is to say, that
if a Filipino word is the concatenation of a root word that ends in "e" or "o" and
a suffix. Then, "e" will change to "i" and "o" will change to "u".

Finally, "e" changes to "i" and "o" changes to "u" if the word is a concatenation
of a root word with itself without a \textit{gitling}. Let
\(s\in\mathcal{F}\) and \(s\models s'\text{\textbackslash}\text{-}s\)
for \(s'\) is the word \(s\) but if \(s\) ends in "e" \(s'\) ends in "i", or if
\(s\) ends in "o" then \(s'\) ends in "u".

\subsubsection{Raw v. Daw (Enclitic Particles); Rin v. Din}

Let \(p = "\,"\) the blank symbol or the symbol containing space. And, we have the
sentence structure \(S = \alpha + p + EP\). The Enclitic Particles (\(\mathbb{EP}\))
is the set \(\mathbb{EP} = \{"\text{raw}", "\text{daw}"\}\) and \(n\) is any \(\alpha\)
is any noun, adjective, verb, or adverb. The usage of "raw" and "daw" is given by:

\[
    \left(\alpha \models \left[\text{a-zñ(ng)A-ZÑ(Ng)}\right]+\left[\text{aeiou}\right] \equiv (\mathbb{V}\cup\mathbb{C}) + \mathbb{V}_\text{lower}\right)
    \Longleftrightarrow EP = \text{"raw"}
\]

Otherwise,

\[
    EP = \text{"daw"}
\]

In other words, if the preceeding word to the enclitic particle ends in vowel,
then the enclitic particle is "raw". If it is a consonant, then it is "daw".

This is the same idea with the adverbs "rin" and "din". If the sentence structure
\(S = \omega + p + (\text{"rin"}|\text{"din"})\). If \(\omega\) ends in a vowel
then the adverb used is "rin". Otherwise, the adverb is "din".

\subsubsection{Ng v. Nang}

\subsubsection{\textit{Gitling} Usage}